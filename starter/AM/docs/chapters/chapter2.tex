\chapter{网格自适应性}

\section{网格自适应加密放粗}

网格自适应可以分为局部加密放粗和全局加密放粗,局部和全局的区别关键在于对初始网格的处理,局部网格剖分是对初始网格的修改,而全局网格剖分会重新生成新的网格。
局部加密放粗可以分为按顶点加密放粗和按单元加密放粗,全局加密放粗可以采用quadtree-octree(四叉树/八叉树)、advancing-front(前沿推进)和Delaunay类型的。
图\ref{fig:2-1}中用两个简单的例子说明按顶点加密放粗和按单元加密放粗,左上分图里红点为顶点,围绕该顶点有两条边,对两条边取中点得到两个新点,并和原有顶点相连,进行了细分,
左下分图里两个红点代表两个单元,将两个单元相邻边取中点,和边对面的顶点相连,分别将两个单元进行了细分得到了四个单元,并且保持协调性。全局加密放粗计算量大于局部,此外还有
hierarchic methods(继承方法)、多重网格、非协调和重叠方法,还需要考虑各向异性。

\begin{figure}[!htbp]
  \centering
  \includegraphics[height=5cm]{fig/2/1.png}
  \caption{按定点加密放粗和按单元加密放粗}
  \label{fig:2-1}
\end{figure}


\section{层网格自适应加密放粗}
