%% Generated by Sphinx.
\def\sphinxdocclass{report}
\documentclass[letterpaper,10pt,english]{sphinxmanual}
\ifdefined\pdfpxdimen
   \let\sphinxpxdimen\pdfpxdimen\else\newdimen\sphinxpxdimen
\fi \sphinxpxdimen=.75bp\relax
\ifdefined\pdfimageresolution
    \pdfimageresolution= \numexpr \dimexpr1in\relax/\sphinxpxdimen\relax
\fi
%% let collapsible pdf bookmarks panel have high depth per default
\PassOptionsToPackage{bookmarksdepth=5}{hyperref}

\PassOptionsToPackage{warn}{textcomp}
\usepackage[utf8]{inputenc}
\ifdefined\DeclareUnicodeCharacter
% support both utf8 and utf8x syntaxes
  \ifdefined\DeclareUnicodeCharacterAsOptional
    \def\sphinxDUC#1{\DeclareUnicodeCharacter{"#1}}
  \else
    \let\sphinxDUC\DeclareUnicodeCharacter
  \fi
  \sphinxDUC{00A0}{\nobreakspace}
  \sphinxDUC{2500}{\sphinxunichar{2500}}
  \sphinxDUC{2502}{\sphinxunichar{2502}}
  \sphinxDUC{2514}{\sphinxunichar{2514}}
  \sphinxDUC{251C}{\sphinxunichar{251C}}
  \sphinxDUC{2572}{\textbackslash}
\fi
\usepackage{cmap}
\usepackage[T1]{fontenc}
\usepackage{amsmath,amssymb,amstext}
\usepackage{babel}



\usepackage{tgtermes}
\usepackage{tgheros}
\renewcommand{\ttdefault}{txtt}



\usepackage[Bjarne]{fncychap}
\usepackage{sphinx}

\fvset{fontsize=auto}
\usepackage{geometry}


% Include hyperref last.
\usepackage{hyperref}
% Fix anchor placement for figures with captions.
\usepackage{hypcap}% it must be loaded after hyperref.
% Set up styles of URL: it should be placed after hyperref.
\urlstyle{same}


\usepackage{sphinxmessages}
\setcounter{tocdepth}{2}


\usepackage{dsfont}
\usepackage{braket}
\usepackage{slashed}
\def\degrees{^\circ}
\def\d{{\rm d}}

\def\L{{\mathcal L}}
\def\H{{\mathcal H}}
\def\M{{\mathcal M}}
\def\matrix{}
\def\fslash#1{#1 \!\!\!/}
\def\F{{\bf F}}
\def\R{{\bf R}}
\def\J{{\bf J}}
\def\x{{\bf x}}
\def\y{{\bf y}}
\def\h{{\rm h}}
\def\a{{\rm a}}
\newcommand{\bfx}{\mbox{\boldmath $x$}}
\newcommand{\bfy}{\mbox{\boldmath $y$}}
\newcommand{\bfz}{\mbox{\boldmath $z$}}
\newcommand{\bfv}{\mbox{\boldmath $v$}}
\newcommand{\bfu}{\mbox{\boldmath $u$}}
\newcommand{\bfF}{\mbox{\boldmath $F$}}
\newcommand{\bfJ}{\mbox{\boldmath $J$}}
\newcommand{\bfU}{\mbox{\boldmath $U$}}
\newcommand{\bfY}{\mbox{\boldmath $Y$}}
\newcommand{\bfR}{\mbox{\boldmath $R$}}
\newcommand{\bfg}{\mbox{\boldmath $g$}}
\newcommand{\bfc}{\mbox{\boldmath $c$}}
\newcommand{\bfxi}{\mbox{\boldmath $\xi$}}


%\def\back{\!\!\!\!\!\!\!\!\!\!}
\def\back{}
\def\col#1#2{\left(\matrix{#1#2}\right)}
\def\row#1#2{\left(\matrix{#1#2}\right)}
\def\mat#1{\begin{pmatrix}#1\end{pmatrix}}
\def\matd#1#2{\left(\matrix{#1\back0\cr0\back#2}\right)}
\def\p#1#2{{\partial#1\over\partial#2}}
\def\cg#1#2#3#4#5#6{({#1},\,{#2},\,{#3},\,{#4}\,|\,{#5},\,{#6})}
\def\half{{\textstyle{1\over2}}}
\def\jsym#1#2#3#4#5#6{\left\{\matrix{
{#1}{#2}{#3}
{#4}{#5}{#6}
}\right\}}
\def\diag{\hbox{diag}}

\font\dsrom=dsrom10
\def\one{\hbox{\dsrom 1}}

\def\res{\mathop{\mathrm{Res}}}

\def\mathnot#1{\text{"$#1$"}}


%See Character Table for cmmib10:
%http://www.math.union.edu/~dpvc/jsmath/download/extra-fonts/cmmib10/cmmib10.html
\font\mib=cmmib10
\def\balpha{\hbox{\mib\char"0B}}
\def\bbeta{\hbox{\mib\char"0C}}
\def\bgamma{\hbox{\mib\char"0D}}
\def\bdelta{\hbox{\mib\char"0E}}
\def\bepsilon{\hbox{\mib\char"0F}}
\def\bzeta{\hbox{\mib\char"10}}
\def\boldeta{\hbox{\mib\char"11}}
\def\btheta{\hbox{\mib\char"12}}
\def\biota{\hbox{\mib\char"13}}
\def\bkappa{\hbox{\mib\char"14}}
\def\blambda{\hbox{\mib\char"15}}
\def\bmu{\hbox{\mib\char"16}}
\def\bnu{\hbox{\mib\char"17}}
\def\bxi{\hbox{\mib\char"18}}
\def\bpi{\hbox{\mib\char"19}}
\def\brho{\hbox{\mib\char"1A}}
\def\bsigma{\hbox{\mib\char"1B}}
\def\btau{\hbox{\mib\char"1C}}
\def\bupsilon{\hbox{\mib\char"1D}}
\def\bphi{\hbox{\mib\char"1E}}
\def\bchi{\hbox{\mib\char"1F}}
\def\bpsi{\hbox{\mib\char"20}}
\def\bomega{\hbox{\mib\char"21}}

\def\bvarepsilon{\hbox{\mib\char"22}}
\def\bvartheta{\hbox{\mib\char"23}}
\def\bvarpi{\hbox{\mib\char"24}}
\def\bvarrho{\hbox{\mib\char"25}}
\def\bvarphi{\hbox{\mib\char"27}}

%how to use:
%$$\alpha\balpha$$
%$$\beta\bbeta$$
%$$\gamma\bgamma$$
%$$\delta\bdelta$$
%$$\epsilon\bepsilon$$
%$$\zeta\bzeta$$
%$$\eta\boldeta$$
%$$\theta\btheta$$
%$$\iota\biota$$
%$$\kappa\bkappa$$
%$$\lambda\blambda$$
%$$\mu\bmu$$
%$$\nu\bnu$$
%$$\xi\bxi$$
%$$\pi\bpi$$
%$$\rho\brho$$
%$$\sigma\bsigma$$
%$$\tau\btau$$
%$$\upsilon\bupsilon$$
%$$\phi\bphi$$
%$$\chi\bchi$$
%$$\psi\bpsi$$
%$$\omega\bomega$$
%
%$$\varepsilon\bvarepsilon$$
%$$\vartheta\bvartheta$$
%$$\varpi\bvarpi$$
%$$\varrho\bvarrho$$
%$$\varphi\bvarphi$$

%small font
\font\mibsmall=cmmib7
\def\bsigmasmall{\hbox{\mibsmall\char"1B}}

\def\Tr{\hbox{Tr}\,}
\def\Arg{\hbox{Arg}}
\def\atan{\hbox{atan}}

\usepackage{CJKutf8}
\AtBeginDocument{\begin{CJK*}{UTF8}{gbsn}}
\AtEndDocument{\end{CJK*}}


\title{Open Digital Twin}
\date{Jun 07, 2022}
\release{0.1}
\author{Jiping Xin}
\newcommand{\sphinxlogo}{\vbox{}}
\renewcommand{\releasename}{Release}
\makeindex
\begin{document}

\pagestyle{empty}
\sphinxmaketitle
\pagestyle{plain}
\sphinxtableofcontents
\pagestyle{normal}
\phantomsection\label{\detokenize{index::doc}}
\sphinxstepscope



\sphinxAtStartPar
欢迎使用开源数字孪生项目!ODT由FENGSim、OpenCAE+和GCGE构成,其中FENGSim包括Cosmic Cube集成开发环境和Airfoil Benchmark架构,Airfoil架构包括FEniCS教程中文版本。OpenCAE+包括OpenCAEPoro和FASP。OpenCAE+和GCGE分别由中国科学院计算数学所张晨松副研究员和谢和虎研究员主持开发。

\sphinxstepscope


\chapter{Cosmic Cube集成开发环境}
\label{\detokenize{src/cosmiccube/main:cosmic-cube}}\label{\detokenize{src/cosmiccube/main::doc}}
\sphinxstepscope


\section{命令获取}
\label{\detokenize{src/cosmiccube/cosmiccube:id1}}\label{\detokenize{src/cosmiccube/cosmiccube::doc}}
\begin{sphinxVerbatim}[commandchars=\\\{\}]
\PYG{g+gp}{\PYGZgt{}\PYGZgt{}\PYGZgt{} }\PYG{o}{/}\PYG{n+nb}{bin}\PYG{o}{/}\PYG{n}{bash} \PYG{o}{\PYGZhy{}}\PYG{n}{c} \PYG{l+s+s2}{\PYGZdq{}}\PYG{l+s+s2}{\PYGZdl{}(curl https://raw.githubusercontent.com/OpenDigitalTwin\PYGZhy{}Dev/OpenDigitalTwin/main/cube)}\PYG{l+s+s2}{\PYGZdq{}}
\end{sphinxVerbatim}

\noindent\sphinxincludegraphics[width=400\sphinxpxdimen]{{cosmiccube}.jpg}

\sphinxstepscope


\chapter{Airfoil CAX架构}
\label{\detokenize{src/airfoil/main:airfoil-cax}}\label{\detokenize{src/airfoil/main::doc}}
\sphinxstepscope


\section{前后处理}
\label{\detokenize{src/airfoil/airfoil_prepost:id1}}\label{\detokenize{src/airfoil/airfoil_prepost::doc}}
\sphinxAtStartPar
ODT\_模块名称\_函数名称


\subsection{图形用户界面设计}
\label{\detokenize{src/airfoil/airfoil_prepost:id2}}

\subsubsection{主界面设计}
\label{\detokenize{src/airfoil/airfoil_prepost:id3}}\begin{enumerate}
\sphinxsetlistlabels{\arabic}{enumi}{enumii}{}{.}%
\item {} 
\sphinxAtStartPar
模仿GIMP

\item {} 
\sphinxAtStartPar
通过菜单和工具栏进入CAD建模、定义物理模型、网格剖分、求解器、可视化五个模块
\begin{itemize}
\item {} 
\sphinxAtStartPar
梳理菜单和工具栏对应的action

\item {} 
\sphinxAtStartPar
梳理和action connection的函数名称和功能

\end{itemize}

\item {} 
\sphinxAtStartPar
主要操作通过浮动工具栏实现

\end{enumerate}


\subsubsection{CAD建模浮动工具栏设计}
\label{\detokenize{src/airfoil/airfoil_prepost:cad}}

\subsubsection{定义物理模型浮动工具栏设计}
\label{\detokenize{src/airfoil/airfoil_prepost:id4}}

\subsubsection{网格剖分浮动工具栏设计}
\label{\detokenize{src/airfoil/airfoil_prepost:id5}}

\subsubsection{求解器浮动工具栏设计}
\label{\detokenize{src/airfoil/airfoil_prepost:id6}}

\subsubsection{可视化浮动工具栏设计}
\label{\detokenize{src/airfoil/airfoil_prepost:id7}}
\sphinxstepscope


\section{求解器}
\label{\detokenize{src/airfoil/airfoil_solver:id1}}\label{\detokenize{src/airfoil/airfoil_solver::doc}}

\subsection{M++}
\label{\detokenize{src/airfoil/airfoil_solver:m}}
\sphinxstepscope


\chapter{OpenCAE+}
\label{\detokenize{src/opencaeplus/main:opencae}}\label{\detokenize{src/opencaeplus/main::doc}}

\section{OpenCAEPoro}
\label{\detokenize{src/opencaeplus/main:opencaeporo}}

\section{FASP}
\label{\detokenize{src/opencaeplus/main:fasp}}
\sphinxstepscope


\chapter{GCGE}
\label{\detokenize{src/gcge/main:gcge}}\label{\detokenize{src/gcge/main::doc}}

\section{GCGE}
\label{\detokenize{src/gcge/main:id1}}
\sphinxstepscope


\chapter{FENGSim}
\label{\detokenize{api/library_root:fengsim}}\label{\detokenize{api/library_root::doc}}

\section{Class Hierarchy}
\label{\detokenize{api/library_root:class-hierarchy}}



\section{File Hierarchy}
\label{\detokenize{api/library_root:file-hierarchy}}



\section{Full API}
\label{\detokenize{api/library_root:full-api}}

\subsection{Classes and Structs}
\label{\detokenize{api/library_root:classes-and-structs}}
\sphinxstepscope


\subsubsection{Class test}
\label{\detokenize{api/classtest:class-test}}\label{\detokenize{api/classtest:exhale-class-classtest}}\label{\detokenize{api/classtest::doc}}\begin{itemize}
\item {} 
\sphinxAtStartPar
Defined in \DUrole{xref,std,std-ref}{file\_include\_test.h}

\end{itemize}


\paragraph{Class Documentation}
\label{\detokenize{api/classtest:class-documentation}}\index{test (C++ class)@\spxentry{test}\spxextra{C++ class}}

\begin{fulllineitems}
\phantomsection\label{\detokenize{api/classtest:_CPPv44test}}
\pysigstartsignatures
\pysigstartmultiline
\pysigline{\phantomsection\label{\detokenize{api/classtest:classtest}}\DUrole{k}{class}\DUrole{w}{  }\sphinxbfcode{\sphinxupquote{\DUrole{n}{test}}}}
\pysigstopmultiline
\pysigstopsignatures
\begin{sphinxuseclass}{breathe-sectiondef}\subsubsection*{Public Functions}
\index{test::test (C++ function)@\spxentry{test::test}\spxextra{C++ function}}

\begin{fulllineitems}
\phantomsection\label{\detokenize{api/classtest:_CPPv4N4test4testEv}}
\pysigstartsignatures
\pysigstartmultiline
\pysiglinewithargsret{\phantomsection\label{\detokenize{api/classtest:classtest_1ab42d5ece712d716b04cb3f686f297a26}}\sphinxbfcode{\sphinxupquote{\DUrole{n}{test}}}}{}{}
\pysigstopmultiline
\pysigstopsignatures
\end{fulllineitems}


\end{sphinxuseclass}
\end{fulllineitems}



\chapter{Indices and tables}
\label{\detokenize{index:indices-and-tables}}\begin{itemize}
\item {} 
\sphinxAtStartPar
\DUrole{xref,std,std-ref}{genindex}

\item {} 
\sphinxAtStartPar
\DUrole{xref,std,std-ref}{search}

\end{itemize}



\renewcommand{\indexname}{Index}
\printindex
\end{document}